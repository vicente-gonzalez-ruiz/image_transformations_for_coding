% Emacs, this is -*-latex-*-
\title{Image transformations for compression}

\author{Vicente González Ruiz}

\maketitle
\tableofcontents

\section{Insights}\begin{itemize}
\tightlist
\item
  Signals can be represented at least in two different domains: the
  signal domain (for example, time in the case of sound of space in the
  case of image) and the frequency domain (in general, in a transform
  domain).
\item
  Transform coding is based on the idea that, in the transform domain,
  most of the energy of the signal can be compacted in a small number
  of (transform) coefficients. This property can be interesting for
  increasing the coding efficiency.
\item
  Scalar quantization of decorrelated transform coefficients is more
  efficient than direct scalar quantization of the
  samples~\cite{vetterli1995wavelets}.
\end{itemize}

\section{Basic coding steps}
\subsection{Encoder}
\begin{enumerate}
\tightlist
\item
  Split \(s\) into blocks of \(B\) samples, if required.
\item
  Transform each block.
\item
  (Optional) Quantize the coefficients.
\item
  Lossless encode the quantized coefficients, performing a minimal
  bit-allocation.
\end{enumerate}

\subsection{Decoder}
\begin{enumerate}
\tightlist
\item
  Decode the coefficients of each block.
\item
  (Optional) ``Dequantize'' the coefficients of each block.
\item
  Inverse-transform each block.
\item
  Join the blocks, if required.
\end{enumerate}

\section{Splitting}
\begin{enumerate}
\tightlist
\item
  Divide \(s=\{s_n\}_{n=0}^{N-1}\) into \(\lceil N/B \rceil\) blocks
  \(\{c_1, \cdots, c_{\lceil N/B \rceil}\}\) of \(B\) samples.
\end{enumerate}

\section{Transform of a block}
\begin{itemize}
\item
  In the forward transform the samples of the block are correlated with
  (the block is compared to) a set of basis functions

  \begin{equation}
    \{C_u\}_{u=0}^{B-1} = \sum_{n=0}^{B-1}a_{u,n}c_n.
    \tag{forward\_transform}
  \end{equation}
\item
  The backward (inverse) transform restores the original samples

  \begin{equation}
    \{c_n\}_{n=0}^{B-1} = \sum_{u=0}^{B-1}b_{n,u}C_u.
    \tag{inverse\_transform}
  \end{equation}
\item
  These equations can be written in matrix form as

  \begin{equation}
    C=Ac
    \tag{forward\_transform\_matrix\_form}
  \end{equation}

  \begin{equation}
    c=BC,
    \tag{inverse\_transform\_matrix\_form}
  \end{equation}

  where \(A\) and \(B\) are matrices, being
  \begin{equation}
    \begin{array}{l}
    [A]_{u,n} = a_{u,n} \\\relax
    [B]_{n,u} = b_{n,u}.
    \end{array}
  \end{equation}

\item
  In transform coding, \(A\) and \(B\) must be inverses of each other
  (\(B=A^{-1}\)), i.e.

  \begin{equation}
    AB = BA = I,
  \end{equation}

  where \(I\) is the identity matrix.
\end{itemize}

\section{A color transform}
\subsection{Luminance and chrominance}
\begin{itemize}
\tightlist
\item
  \href{https://en.wikipedia.org/wiki/Chrominance}{Chrominance} (or
  chroma) is the signal used in video systems to convey the color
  information of the picture or a video. It was defined to add the color
  signal to the black and white one in analog TV. Thus, the same signal,
  composed by two different subsignals: Y and UV that can be isolated by
  filtering, was compatible with both, black and white (which only used
  Y) and color ones (that used
  \href{https://en.wikipedia.org/wiki/YUV}{YUV}).
\end{itemize}

\begin{equation}
    \left(
      \begin{array}{c}
        \text{Y}\\
        \text{U}\\
        \text{V}
      \end{array}
    \right) =
    \left(
      \begin{array}{rrr}
          0,299 & 0,587 & 0,144 \\
          -0.14713 & -0.28886 &  0.436 \\
          0.615   & -0.51499 & -0.10001
      \end{array}
    \right)
    \left(
      \begin{array}{c}
        \text{R}\\
        \text{G}\\
        \text{B}
      \end{array}
    \right)
\end{equation}

\begin{equation}
    \left(
      \begin{array}{c}
        \text{R}\\
        \text{G}\\
        \text{B}
      \end{array}
    \right) =
    \left(
      \begin{array}{rrr}
          1 &  0       &  1.13983 \\
          1 & -0.39465 & -0.58060 \\
          1 &  2.03211 &  0
      \end{array}
    \right)
    \left(
      \begin{array}{c}
        \text{Y}\\
        \text{U}\\
        \text{V}
      \end{array}
    \right)
\end{equation}

\begin{itemize}
\item
  Later, in digital video, the YUV color domain was called the
  \href{https://en.wikipedia.org/wiki/YCbCr}{YCrCb color domain}.
\item
  Used, for example, in \href{https://en.wikipedia.org/wiki/JPEG}{JPEG}.
\end{itemize}

\subsection{Spectral (color) redundancy}
\begin{itemize}
\tightlist
\item
  \href{https://en.wikipedia.org/wiki/RGB_color_model}{\(\text{RGB}\)
  domain} is more redundant than the
  \href{https://en.wikipedia.org/wiki/YUV}{\(\text{YUV}\) domain}:
\end{itemize}

%\iframe{1200}{1200}{https://nbviewer.jupyter.org/github/vicente-gonzalez-ruiz/image_transformations_for_coding/blob/master/color_redundancy.ipynb}
\href{https://nbviewer.jupyter.org/github/vicente-gonzalez-ruiz/image_transformations_for_coding/blob/master/color_redundancy.ipynb}{Color
  redundancy~\cite{barr__image, gouillart__scikit, solem__programming}}.

\section{Chrominance subsampling}
\href{https://en.wikipedia.org/wiki/Chroma_subsampling}{The human visual
system is more sensitive to the luma (Y) than to the chroma (UV)}. This
means than the chroma can be subsampled without a significant loss of
quality in the images.

\svg{color_subsampling}{600}

%\iframe{1200}{1200}{https://nbviewer.jupyter.org/github/vicente-gonzalez-ruiz/image_transformations_for_coding/blob/master/chroma_subsampling.ipynb}
\href{https://nbviewer.jupyter.org/github/vicente-gonzalez-ruiz/image_transformations_for_coding/blob/master/chroma_subsampling.ipynb}{Chroma subsampling}.

\section{Orthogonal transform}
\begin{itemize}
\tightlist
\item
  The rows of \(A\) (\(a_{k,*}\)) are refered to as the \emph{basis
  vectors} of the transform, and should form an \emph{orthogonal} basis
  set in order to provide maximum energy compactation. The rows can be
  also seen as the coefficients of \(B\) filters, being the first one
  (\(i=0\)) the ``low-pass'' one, which will produce the DC coefficient,
  and the rest (\(i\geq 1\)) the ``high-pass'' filters, which will
  generate the AC (Alternating Current) coeffs. These \(B\) filters form
  a filter-bank where the overlapping between the frequency response of
  the filters should be as small as possible if we want maximum energy
  compaction.
\end{itemize}

\subsection{Orthonormal transform}
\begin{itemize}
\item
  If the basis vectors of a orthogonal transform are unit vectors, the
  transform is said orthonormal.
\item
  For orthonormal transforms, it holds that

  \begin{equation}
    A^{-1} = A^T.
  \end{equation}

  Therefore, the pair of transforms refered by Eqs.
  (forward\_transform\_matrix\_form) and
  (inverse\_transform\_matrix\_form) can be written as

  \begin{equation}
    \begin{array}{l}
      C=Ac \\
      c=A^TC.
    \end{array}
  \end{equation}
\end{itemize}

\section{Signal energy}
\begin{equation}
  ||s||^2 = \sum_{n=0}^{B-1}s_n^2.
\end{equation}

\subsection{Energy conservation}
\begin{itemize}
\item
  if \(A\) is orthonormal (also called \emph{unitary}), the energy of
  the transformed signal is the equal to the original one:

  \begin{equation}
     ||C||^2 = ||c||^2.
  \end{equation}
\end{itemize}

\subsection{Proof}
\begin{equation}
  ||C||^2 = C^TC = (Ac)^TAc = c^TA^TAc = c^TIc = c^Tc = ||c||^2.
\end{equation}

\section{\href{https://en.wikipedia.org/wiki/Expected_value}{Expected value}}
The expected value \(\operatorname{E}[X]\) of a random variable \(X\),
intuitively, is the long-run average value of repetitions of the
experiment it represents. Let \(X\) be a random variable with a finite
number of finite outcomes \(X_1\), \(X_2\), \(\cdots\), \(X_n\)
occurring with probabilities \(p_1\), \(p_2\), \(\cdots\), \(p_k\),
respectively. The expected value (or \emph{expectation}) of \(X\) is
defined as

\begin{equation}
  \operatorname{E}[X] = \sum_{i=1}^n X_ip_i.
\end{equation}

\section{\href{https://en.wikipedia.org/wiki/Variance}{Variance}}
The variance of a random variable \(X\) is the expected value of the
squared deviation from the mean of \(X\):

\begin{equation}
  \sigma_X=\operatorname{Var}(X) = \operatorname{E}\left[(X - \operatorname{E}[X])^2 \right] = \operatorname{E}\left[(X - \operatorname{E}[X])(X - \operatorname{E}[X]) \right] = \operatorname{E}\left[X^2 \right] - \operatorname{E}[X]^2.
\end{equation}

\subsection{\href{https://en.wikipedia.org/wiki/Covariance}{Covariance}}
The covariance \(\operatorname{cov}(X,Y)\) is a measure of the joint
variability of two random variables \(X\), \(Y\), defined as:

\begin{equation}
  \operatorname{cov}(X,Y) = \operatorname{E}{\big[(X - \operatorname{E}[X])(Y - \operatorname{E}[Y])\big]},
\end{equation}

\section{\href{https://en.wikipedia.org/wiki/Covariance_matrix}{Covariance
matrix}}
A covariance matrix \(\Sigma_Z\) is a matrix whose element in the \(i\),
\(j\) position is the covariance between the \(i\)-th and \(j\)-th
elements of a random vector \(Z\) (a collection of random variables
\(Z_i\)):

\begin{equation}
  \Sigma_Z =
    \begin{bmatrix}
       \operatorname{E}[(Z_1 - \mu_1)(Z_1 - \mu_1)] & \operatorname{E}[(Z_1 - \mu_1)(Z_2 - \mu_2)] & \cdots & \operatorname{E}[(Z_1 - \mu_1)(Z_n - \mu_n)] \\ \\
       \operatorname{E}[(Z_2 - \mu_2)(Z_1 - \mu_1)] & \operatorname{E}[(Z_2 - \mu_2)(Z_2 - \mu_2)] & \cdots & \operatorname{E}[(Z_2 - \mu_2)(Z_n - \mu_n)] \\ \\
       \vdots & \vdots & \ddots & \vdots \\ \\
       \operatorname{E}[(Z_n - \mu_n)(Z_1 - \mu_1)] & \operatorname{E}[(Z_n - \mu_n)(Z_2 - \mu_2)] & \cdots & \operatorname{E}[(Z_n - \mu_n)(Z_n - \mu_n)]
    \end{bmatrix} =
    \operatorname{E}
    \left[
       (Z - \operatorname{E}[Z])
       (Z - \operatorname{E}[Z])^{\rm T}
    \right].
\end{equation} where \begin{equation}
  \mu_i = \operatorname{E}(Z_i),
\end{equation} is the expected value of the \(i\)-th entry in the vector
\(Z\).

\section{Covariance matrix of a block-based transform}
\begin{equation}
  \Sigma_S = \operatorname{E} \left[
       (S - \operatorname{E}[S])
       (S - \operatorname{E}[S])^{\rm T}
    \right] = \operatorname{E} \left[
       A(s - \operatorname{E}[s])
       (s - \operatorname{E}[s])^{\rm T}A^{\rm T}
    \right] = 
    A \operatorname{E} \left[
       (s - \operatorname{E}[s])
       (s - \operatorname{E}[s])^{\rm T}
    \right] A^{\rm T} = 
    A\Sigma_sA^{\rm T}.
\end{equation}

\section{\href{https://en.wikipedia.org/wiki/Correlation_and_dependence}{Correlation}}
The most familiar measure of dependence between two random variables
\(X\) and \(Y\) is the Pearson product-moment correlation coefficient,
or ``Pearson's correlation coefficient'', commonly called simply ``the
correlation coefficient''. It is obtained by dividing the covariance of
the two variables by the product of their standard deviations.

\begin{equation}
  \rho_{X,Y}=\mathrm{corr}(X,Y)={\mathrm{cov}(X,Y) \over \sigma_X \sigma_Y} ={\operatorname{E}[(X-\mu_X)(Y-\mu_Y)] \over \sigma_X\sigma_Y}
\end{equation}

\section{\href{https://en.wikipedia.org/wiki/Autocorrelation}{Autocorrelation}}
Autocorrelation, also known as serial correlation, is the correlation of
a signal \(s[t]\) with a delayed copy of itself as a function of delay
\(s[t+\tau]\).

\begin{equation}
  \rho_s[\tau]=\rho_{s[t],s[t+\tau]}={\operatorname{E}[(s[t]-\mu)(s[t+\tau]-\mu)] \over \sigma^2}
\end{equation}

where \(\mu=\operatorname{E}(s[t])=\operatorname{E}(s[t+\tau])\) and
\(\sigma=\sigma_{s[t]}=\sigma_{s[t+\tau]}\).

\section{Autocorrelation matrix}
The autocorrelation matrix of a random process \(X\) is the matrix
\([R]\) defined by

\begin{equation}
  [R]_{i,j} = \rho_X[|i-j|].
\end{equation}

\section{\href{https://en.wikipedia.org/wiki/Eigenvalues_and_eigenvectors}{Eigenvalue and eigenvector}}
In linear algebra, an eigenvector or characteristic vector
\(\overrightarrow{v}\) of a linear transformation \(T()\) is a non-zero
vector that changes by only a scalar factor \(\lambda\) (known as the
eigenvalue, characteristic value, or characteristic root of
\(\overrightarrow{v}\)) when that linear transformation is applied to
it:

\begin{equation}
  {\displaystyle T(\overrightarrow{v} )=\lambda \overrightarrow{v}}.
\end{equation}

When \(T()\) can be expressed by a matrix \(X\), we get

\begin{equation}
  A\overrightarrow{v} =\lambda \overrightarrow{v}.
\end{equation}

\section{Coding gain}
\begin{itemize}
\item
  The coding gain measures the compaction level of the transform, which
  is defined as

  \begin{equation}
    G=\frac{\frac{1}{B}\displaystyle\sum_{u=0}^{B-1}\sigma_{C_u}^2}{\sqrt[B]{\displaystyle\prod_{u=0}^{B-1}\sigma_{C_u}^2}},
  \end{equation}

  where \(\sigma_{C_u}^2\) is the variance of coeff \(C_u\).
\end{itemize}

\section{Karhunen-Loéve transform (KLT)}
\begin{itemize}
\item
  For the KLT, the rows of \(A\) (the basis of the forward transform)
  are the eigenvectors of the (unnormalized) autocorrelation matrix
  \([R]\) of the signal \(s\), where

  \begin{equation}
    [R]_{i,j} = \operatorname{E}\big(s_ns_{n+|i-j|}\big).
  \end{equation}
\item
  It can be proven that KLT minimizes
  \(\sqrt[B]{\prod_{u=0}^{B-1}\sigma_{C_u}^2}\), and therefore, it
  provides the maximum coding gain. Unfortunately, the basis fuctions of
  the KLT depends on \(s\). If \(S\) is
  non-\href{https://en.wikipedia.org/wiki/Stationary_process}{stationary},
  the autocorrelation matrix (or the basis) must be sent to the decoder
  (to run the inverse transform) as side information. However, if
  \(B=2\), the KLT is

  \begin{equation}
    A_{\text{2-KLT}} = \frac{1}{\sqrt{2}}
    \left[
      \begin{array}{cc}
        1 & 1 \\
        1 & -1
      \end{array}
    \right]
  \end{equation}

  for all signals.
\end{itemize}

\section{Discrete cosine transform (DCT))}
\hypertarget{definition}{%
\subsubsection{Definition}\label{definition}}

\begin{itemize}
\item
  The forward (direct) transform is

  \begin{equation}
    S_u = \frac{\sqrt{2}}{\sqrt{N}}
    K(u)\sum_{n=0}^{N-1} s_n\cos\frac{(2n+1)\pi u}{2n},
  \end{equation}

  and the backward (inverse) transform is

  \begin{equation}
    s_n = \frac{\sqrt{2}}{\sqrt{N}}
    \sum_{u=0}^{N-1} K(u)S_u\cos\frac{(2n+1)\pi u}{2n},
  \end{equation}

  where \(N\) is the number of pixels, and \(s_n\) denotes the \(n\)-th
  pixel of the image \(s\), and

  \begin{equation}
    K(u) =
    \left\{
      \begin{array}{ll}
      \frac{1}{\sqrt{2}} & \text{si}~u=0\\
        1 & \text{if}~u>0.
      \end{array}
      \right.
  \end{equation}
\end{itemize}

\subsection{Properties fo DCT}
\begin{enumerate}
\tightlist
\item
  \textbf{Separable}: the \(D\)-dimensional DCT can be computed using
  the \(1\)D DCT in each possible dimension.
\item
  In general, \textbf{high energy compaction}: a small number of DCT
  coefficients can reconstruct with a reasonable accuracy the original
  signal.
\item
  \textbf{Unitary}: the energy of the DCT coefficients is proportional
  to the energy of the samples.
\item
  \textbf{Orthonormality}: DCT basis are orthonormal (orthogonal +
  unitary) and therefore, DCT coefficients are uncorrelated.
\end{enumerate}

%\iframe{1200}{1200}{https://nbviewer.jupyter.org/github/vicente-gonzalez-ruiz/image_transformations_for_coding/blob/master/DCT.ipynb}
\href{https://nbviewer.jupyter.org/github/vicente-gonzalez-ruiz/image_transformations_for_coding/blob/master/DCT.ipynb}{DCT}.

\section{Dyadic discrete wavelet transform (DWT)}
Key \href{https://en.wikipedia.org/wiki/Discrete_wavelet_transform}{features}:

\begin{enumerate}
\tightlist
\item
  \textbf{High spectral compaction}, specially when transient signals
  are present.
\item
  \textbf{Multiresolution representation}: it is easy to recover a
  reduced version of the original image if only a sub-set of the
  coefficients is proccesed.
\end{enumerate}

\section{Filters bank implementation}
Where: \begin{equation}
  s = (\uparrow^2(L)*{\text s}_L) + (\uparrow^2(H)*{\text s}_H)
\end{equation} and \begin{equation}
  \begin{array}{rcl}
    L & = & \downarrow^2(s*{\text a}_L) \\
    H & = & \downarrow^2(s*{\text a}_H).
  \end{array}
\end{equation}

Comments:

\begin{enumerate}
\item
  \({\text a}_L\) and \({\text a}_H\) are the transfer function (the
  transfer function of a filter is the response of that filter to the
  unitary impulse function (Dirac's delta)) of a low-pass filter and
  high-pass filter, respectively, that have been designed to be
  complementary (ideally, in \(L\) we only found the frequencies of
  \(s\) that are not in \(H\), and viceversa). When this is true, it is
  said the we are using a perfect-reconstruction quadrature-mirror
  filter-bank and the DWT is \emph{biorthogonal}.
\item
  In the wavelet theory, \({\text s}_L\) is named the \emph{scale
  function} and \({\text s}_H\) the \emph{mother function} or
  \emph{wavelet basis function}. The coefficients of \(L\) are also
  known as the \emph{scale coeffients} and the coeffcientes of \(H\) the
  \emph{wavelet coefficients}~\cite{sovic2012signal}.
\item
  \(\downarrow^2(\cdot)\) and \(\uparrow^2(\cdot)\) donote the
  subsampling and oversampling operations:
\end{enumerate}

\begin{equation}
    (\downarrow^2(s))_i = s_{2i}
\end{equation}

and

\begin{equation}
    (\uparrow^2(s))_i =
  \left\{
  \begin{array}{ll}
    s_{i/2} & \text{if $i$ if even} \\
    0 & \text{otherwise}.
  \end{array}
  \right.
  \end{equation}

where \(s_i\) if the \(i\)-th sample of \(s\).

\begin{enumerate}
\setcounter{enumi}{3}
\item
  \(*\) is the convolution operator.
\item
  Notice that half of the filtered samples are wasted.
\end{enumerate}

\section{Lifting implementation~\cite{sweldens1997building}}
\svg{lifting}{1000}

Comments:

\begin{enumerate}
\tightlist
\item
  \begin{equation}
  H_i = s_{2i+1} - {\cal P}(\{s_{2i}\})_i
  \tag{PredictionStep}
  \label{eq:PredictionStep}
  \end{equation}
\end{enumerate}

\begin{equation}
  L_i = s_{2i} + \{{\cal U}(H)\}_i
  \tag{UpdateStep}
  \label{eq:UpdateStep}
\end{equation}

\begin{enumerate}
\setcounter{enumi}{1}
\tightlist
\item
  Subsampled signals \(\{s_{2i}\}\) and \(\{s_{2i+1}\}\) can been
  computed by using
\end{enumerate}

\begin{equation*}
   \{s_{2i+1}\} = \downarrow^2(Z^{-1}(s))
\end{equation*}

and

\begin{equation*}
   \{s_{2i}\} = \downarrow^2(s),
\end{equation*}

where \(Z^{-1}\) represents the one sample delay function.

\begin{enumerate}
\setcounter{enumi}{2}
\tightlist
\item
  \(H\) has tipically less energy and variance and entropy than
  \(\{s_{2i+1}\}\).
\item
  \(L\) has less aliasing than \(\{s_{2i}\}\) (notice that \(L\) has not
  been low-pass filtered before subsampling it).
\end{enumerate}

\section{$T$-levels 1D-DWT}
\svg{n_levels_dwt1d}{1000}

\section{Subband indepency~\cite{vetterli1995wavelets}}
While the subbands are only independent if the input is a Gaussian
random variable and the filters decorrelate the subbands (the filters
are ideal), the independence assumption is ofte made because it makes
he system simpler.

\section{Statistics of the subbands~\cite{vetterli1995wavelets}}
The PDF of the coefficients of the high-frequency subbands peaks in
zero and falls off very rapidly. While is it often modeled as a
Laplacian distribution, it is actually falling off faster. It is more
adequately fitted with a generalized Gaussian PDF with faster decay
than the Laplacian PDF.

\section{Subband quantization~\cite{vetterli1995wavelets}}
Besides the low band compression, which uses known image coding
methods, the bulk of the compression is obtained by appropiate
quantization of the high bands. The
following \mylink{quantization}{quantizers} are typically used:
\begin{enumerate}
\item Lloyd quantizers fitted to the PDF of the particular subband.
\item Deadzone uniform quantizers, since they tend to eliminate what
  is essentially noise.
\end{enumerate}
Because entropy coding is used after quantization, uniform quantizers
are nearly optimal. Note that VQ could be used in the subbands, but
its complexity is usually not worthwhile since there is little
dependence between coefficients anyway.

\section{2D-DWT}    
\begin{itemize}
\tightlist
\item
  The one-dimensional (1D) DWT is a separable transform. Therefore,
  the 2D DWT can be computed applying the DWT to all the rows of an
  image and next, to all the columns, or viceversa.
\end{itemize}

\svg{2D-DWT}{800}

\begin{itemize}
\tightlist
\item
  The contribution of a coefficient of a subband \(b\) is determined
  by the DWT basis fuction \({s_H}^b\) asociated to that coefficient,
  which can be empirically determined by applying the inverse DWT to
  the Dirac Impulse function localized in that coefficient (notice
  that \({s_H}^b\) does not depend on the coefficient because we are
  supposing that all the coefficients of a subband have the same
  contribution, the same basis
  fuction)~\cite{rabbani2009jpeg}. Therefore, the L\(_2\)-norm for the
  subband \(b\) is computed as the energy of a basis function of that
  subband as
\end{itemize}

\begin{equation}
  E({{\text s}_H}^b) = \sum_i{|{{\text s}_H}^b_i|}^2.
\end{equation}

In the case of the 5/3-DWT, the L\(_2\)-norms of the DWT subbands are:
\svg{factores_5_3_L2_norm}{400}

\section{Haar filters~\cite{haar1910theorie}}
The \(i\)-th sample of the low-frequency subband is computed (using a
filter plus subsampling) as

\begin{equation}
  L_i=\frac{s_{2i}+s_{2i+1}}{2},
  \tag{HaarL}
  \label{eq:Haar_A-LPF}
\end{equation}

and the \(i\)-th sample of the high-frequency subband as

\begin{equation}
  H_i=s_{2i+1}-s_{2i}.
  \tag{HaarH}
  \label{eq:Haar_A-HPF}
\end{equation}

If Lifting is used,

\begin{equation}
  L_i=s_{2i}+\frac{H_i}{2}.
  \tag{HaarLLifted}
  \label{eq:Haar_A-LPF-lifting}
\end{equation}

Notice that \(H_i=0\) if \(s_{2i+1}=s_{2i}\), therefore, the Haar
transform is good to encode constant signals. The notation X/Y indicates
the length (taps or number of coefficients) of the low-pass and the
high-pass (convolution) filters of the filter bank implementation (not
Lifting), respectively.

%\iframe{1200}{1200}{https://nbviewer.jupyter.org/github/vicente-gonzalez-ruiz/image_transformations_for_coding/blob/master/Haar_2d_basis.ipynb}
\href{https://nbviewer.jupyter.org/github/vicente-gonzalez-ruiz/image_transformations_for_coding/blob/master/Haar_2d_basis.ipynb}{Haar basis}.

\section{Linear (5/3) filters~\cite{sweldens1997building}}
The $i$-th sample of the low-frequency subband (using a filter-bank implementation) is

\begin{equation}
  L_i=-\frac{1}{8}s_{2i-2}+\frac{1}{4}s_{2i-1}+\frac{3}{4}s_{2i}
  +\frac{1}{4}s_{2i+1}-\frac{1}{8}s_{2i+2}
  \tag{5/3L}
  \label{eq:Lineal_A-LPF}
\end{equation}

and the $i$-th sample of the high-frequency signal is computed by

\begin{equation}
  H_i=s_{2i+1}-\frac{s_{2i}+s_{2i+2}}{2},
  \tag{5/3H}
  \label{eq:Lineal_A-HPF}
\end{equation}

that, if we use Lifting, it can be also computed using less operations by

\begin{equation}
  L_i=s_{2i}+\frac{H_{i-1}+H_i}{4}.
  \tag{5/3LLifted}
  \label{eq:Lineal_A-LPF_lifting}
\end{equation}

Notice that $H_i=0$ if $s_{2i+1}=(s_{2i}+s_{2i+2})/2$. Therefore, the 5/3 transform is suitable to encode lineally piece-wised signals.

\href{https://nbviewer.jupyter.org/github/vicente-gonzalez-ruiz/image_transformations_for_coding/blob/master/linear_2d_basis.ipynb}{Linear (5/3) basis}.


\section{Orthogonal, orthonormal, and biorthogonal transforms}
%{{{

In signal processing\footnote{In mathemathics the definition of
  orthogonality refers to characteristics such as the basis of the
  transform forms an orthogonal espace, where it is impossible to
  represent one of the basis as a linear combination of the rest of
  basis (or in other words, if the inner product is zero).}, a
transform (such as the
\href{https://en.wikipedia.org/wiki/Discrete_cosine_transform}{Discrete
  Cosine Transform}, the
\href{https://en.wikipedia.org/wiki/Hadamard_transform}{Walsh-Hadamard
  Transform} or the
\href{https://en.wikipedia.org/wiki/Karhunen%E2%80%93Lo%C3%A8ve_theorem}{Karhunen-Loève
  Transform}) is orthogonal when the coefficients generated by the
transform are uncorrelated (there is no way to infer one coefficient
from another).

If the \href{https://en.wikipedia.org/wiki/Norm_(mathematics)}{norm}
of all the basis of an orthogonal transform is one, then the transform
is said
\href{https://en.wikipedia.org/wiki/Orthonormal_basis}{orthonormal}. Orthonormal
transforms are interesting because of their:
\begin{enumerate}
\item
  \textbf{\href{https://en.wikipedia.org/wiki/Energy_(signal_processing)}{Energy}
    preservation}: The energy of the output is the same than the
  energy of the input. This means that, for example, a quantization
  error produced in a coefficient of the transform will generate the
  same quantization error at the output (the complete signal) of the
  inverse transform. The same holds by the forward transform.
\item \textbf{Implementation}: The transform matrix of the
  inverse transform is the transpose of the forward transform. In
  orthogonal transforms, the transform matrix of the inverse transform
  is the inverse of the transform matrix of the forward transform.
\end{enumerate}

Biorthogonal transforms (and in particular,
\href{https://en.wikipedia.org/wiki/Biorthogonal_wavelet}{biorthogonal
  wavelets}) do no satisfy any of these features: they are not energy
preserving (this can be also observed because the frequency-domain
responses of the analysis and synthesis filters are not symmetric),
and there is not an algebraic way (matrix transposition/inversion) to
compute the backward transform from the forward one, and
viceversa. This, that can be considered as a drawback, gives an extra
degree of freedom to design the analysis and the synthesis filters
(whose only requirement is that transform pair to be reversible),
providing in general the possibility of using more sofisticated
filters such as those based on non-linear filtering, as for example,
those that use motion estimation algorithms.

In general, each subband $b$ of a decomposition generated by a
biorthogonal 2D-DWT transform have a different subband gain
$\alpha_b$. Usually, the lower the frequency of the subband, the
higher the gain. Notice also, that these gains also are different for
each transform.

Subband gains are important in lossy signal compression because they
quantify the relative importance of the wavelet coefficients of the
different subbands when we introduce distortion in the wavelet
domain. Thus, for example, if we decide to quantize a wavelet
coefficient, the amount of distortion that we are generating in the
signal domain will depend on the subband where that coefficient is
localized. In general, low-frequency coefficients are more
``important'' that high-frequency ones.

To compute the subband gains we have two options:
\begin{enumerate}
\item The algebraic way. We will need the expressions of the four
  filters (two anaylsis filters and two synthesis filters) and deduce
  the gains.
\item The algorithmic way. We will need to compute the energy of the
  impulse response of the inverse transform, when we apply such
  impulse to each one of the subbands of the decomposition. Supposing
  that, after appliying the DWT to an image, the coefficients of the
  subband HH are the least energetic with an energy $x$, the subband
  gain for subband $b$ is computed as
  \begin{equation}
    \alpha_b = E_b/x,
  \end{equation}
  where $E_b$ is the energy of the reconstruction when the
  impulse signal is localized at $b$. Notice that all the gains should
  be larger than one.
\end{enumerate}

This computation can be also applied to MCDWT, by computing the subband
gains as a function of the number $T$ of temporal decompositions.

%}}}

\section{Quantization in the transform domain}
\begin{itemize}
\tightlist
\item
  If the transform is orthogonal, by definition coeffs \(S[k]\) are
  uncorrelated. Therefore, a scalar quantizer can performs an optimal
  quantization.
\end{itemize}

\section{Bit-planes progression}
\svg{bit-plane-trans}{800}

\subsection{Bit allocation (bit-rate control)}
\begin{itemize}
\item
  In lossless coding, coeffs \(S_u\) are directly encoded using some
  text compression algorithm or a combination of them.
\item
  However, in most situations, a lossy compression is needed and in this
  case, a transform coder must determine, given a maximum number of bits
  \(\overline{R}\) (which is defined by the compression ratio selected
  by the user), the number of bits \(R(u)\) used by the quantizer for
  each coeff \(S_k\).
\end{itemize}

\section{Bit allocation based on minimizing the quantization error}
\begin{itemize}
\tightlist
\item
  In unitary transforms, as a consequence of the energy preserving
  property, an uniform quantization (i.e.~the dividing each coeff
  \(S_u\) by the same quantization step) should provide optimal bit
  allocation if we want to minimize the quantization error (the
  distortion) in the recostructed signal \(s\).
\end{itemize}

\section{Bit allocation based on minimizing the variance of the quantization error}
\begin{itemize}
\item
  Lets assume that the variance of the coeffs, defined as

  \begin{equation}
    \sigma_{S_u}^2 = \text{E}\big( (S_u - \overline{S})^2\big)
   \end{equation}

  (where

  \begin{equation}
     \overline{S} = \text{E}(S) = \frac{1}{B}\sum_{u=0}^{B-1} S_u
   \end{equation}

  corresponds to the amount of information provided by each coeff.
  Therefore, coeffs with high variance should be assigned more bits and
  viceversa.
\item
  Lets define

  \begin{equation}
    {\overline{R}} = \frac{1}{B}\sum_{u=0}^{B-1}R(u)
    \tag{$\overline{R}$}
  \end{equation}

  as the (target) average number of bits/coeff, where \(R(u)\) is the
  number of bits assigned to coeff \(S_u\).
\item
  If the mean square error is as a measure of distortion, the variance
  of the distortion generated by the quantization of a coeff \(S_u\)
  \href{http://cdn.intechopen.com/pdfs/16267/InTech-Rate_control_in_video_coding.pdf}{can
  be modeled} by

  \begin{equation}
    \sigma_{S_u-\tilde{S}_u}^2=\alpha_{u}2^{-2R(u)}\sigma_{S_u}^2,
  \end{equation}

  where \(\alpha_{u}\) depends on the frequency \(u\) and the quantizer.
\end{itemize}

\href{https://nbviewer.jupyter.org/github/vicente-gonzalez-ruiz/image_transformations_for_coding/blob/master/DR_model.ipynb}{DR\_model notebook}.
    
\begin{itemize}
\item
  Assuming an additive distorion metric, the total distortion variance
  for \(R_k\) bits/coeff is given by

  \begin{equation}
    D = \sigma_{S-\tilde{S}}^2 = \sum_{u=0}^{B-1} \sigma_{S_u-\tilde{S}_u}^2 = \sum_{u=0}^{B-1}\alpha_u 2^{-2R(u)}\sigma_{S_u}^2 = \alpha\sum_{u=0}^{B-1}2^{-2R(u)}\sigma_{S_u}^2
    \tag{$D$}
  \end{equation},

  supposing that \(\alpha_u = \alpha\) is constant for all coeffs (a
  valid supposition for unitary transforms because the quantization
  error generated in each coeff should be the same if an uniform
  quantizer is used).
\item
  The objective of the bit-allocation process is to find the
  \(\{R(u)\}_{u=0}^{B-1}\) so that minimize \(D\) subject to constraint
  \(\overline{R}\):

  \begin{equation}
    \underset{\{R(u)\}_{u=0}^{B-1}}{\operatorname{arg min}} D, \text{s.t.}~{\overline{R}}. 
  \end{equation}
\item
  This is an optimization problem that can be solved using
  \href{https://en.wikipedia.org/wiki/Lagrange_multiplier}{Lagrange
  multipliers} (note: the following development is not the ``standard''
  way of using Lagrenge multipliear, but it is equivalent).
\item
  Lets define the Lagrangian functional

  \begin{equation}
    J = D - \lambda\Big( \overline{R} - \frac{1}{B}\sum_{u=0}^{B-1}R(u) \Big)= \alpha \sum_{u=0}^{B-1} 2^{-2R(u)}\sigma_{S_u}^2 - \lambda \Big( \overline{R} - \frac{1}{B}\sum_{u=0}^{B-1}R(u) \Big),
  \end{equation}
\end{itemize}

which taking

\begin{equation}
    \frac{\partial J}{\partial R(u)} = 0
  \end{equation}

produces that

\begin{equation}
    R(u) = \frac{1}{2}\log_2\big( 2\alpha\ln 2\sigma_{S_u}^2 \big) - \frac{1}{2}\log_2\lambda.
    \tag{$R(u)$}
  \end{equation}

\begin{itemize}
\item
  Substituting \(R(u)\) in Eq. (\(\overline{R}\)), we get that

  \begin{equation}
    \overline{R} = \frac{1}{B}\sum_{u=0}^{B-1}\frac{1}{2}\log_2\big( 2\alpha\ln 2\sigma_{S_u}^2 \big) - \frac{1}{2}\log_2\lambda.
  \end{equation}
\item
  Operating

  \begin{equation}
    \lambda = \prod_{u=0}^{B-1}\sqrt[B]{2\alpha\ln 2\sigma_{S_u}^2} - 2^{-2\overline{R}}.
  \end{equation}
\item
  Substituting \(\lambda\) in Eq. (\(R(u)\)), we obtain the optimal
  number of bits for each coeff

  \begin{equation}
    R(u) = \overline{R} + \frac{1}{2}\log_2\frac{\sigma_{S_u}^2}{\displaystyle\prod_{u=0}^{B-1}\sqrt[B]{\sigma_{S_u}^2}}.
  \end{equation}

  which minimizes the variance of the quantization error. Notice that
  this value depends proportionally on \(\overline{R}\) (the target
  average bits/coeff), logaritmically on \(\sigma_{S_u^2}\) (the
  variance of the coeff) and log-inversely on the geometric mean of the
  variances of all coeffs.
\end{itemize}

\section{Encoding}
\begin{itemize}
\tightlist
\item
  For DCT, usually ZigZag-RLE followed by 0-order entropy coding.
\item
  For DWT, tree coding or block-based coding.
\end{itemize}

\section{Code-stream orderings and scalabilities}
\begin{itemize}
\tightlist
\item
  The order in which the DWT coefficients are decoded determines the type of
  scalability (example with 2 qualities and 3 resolutions):
\end{itemize}

\svg{orderers-and-scalabilities}{800}

\bibliography{DWT,Python}
